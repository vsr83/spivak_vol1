\documentclass [12pt, a4paper] {article}

%Babel tekee kuulemma jotain typer��, joten seuraava on v�ltt�m�t�nt�.
\usepackage[all, knot] {xy}
\let\oldxy\xy\def\xy{\begingroup\catcode`\"12\oldxy}
\let\oldendxy\endxy\def\endxy{\oldendxy\endgroup} 
\xyoption {arc}
\usepackage [latin1] {inputenc}
\usepackage [T1] {fontenc}
\usepackage [english] {babel}
\usepackage [dvips] {graphicx}
\usepackage {amsfonts}
\usepackage {amssymb}
\usepackage {amsmath}
\usepackage {verbatim}
\usepackage {fancyhdr}
\usepackage {psfrag}
\usepackage {listings}
\usepackage {eso-pic}

%\usepackage[urw-garamond]{mathdesign}
\usepackage[small,bf]{caption}

\newcommand{\BackgroundPic}{
%  \put(0,0){
%    \parbox[b][\paperheight]{\paperwidth}{
%      \vfill
%      \centering
%      \includegraphics[width=\paperwidth, height=\paperheight, keepaspectratio%]{
%kansilehti.ps}
%    }
%  }
%  \textbf{SMG-5156 - Electromagnetic Modelling I}\\
%  \textbf{Exercise 1 - Problem 4}\\
%  \textbf{Ville R�is�nen}
%\end {flushleft}

%  \put(42, 810){\textbf{SMG-5156 - Electromagnetic Modelling I}}
%  \put(42, 797){\textbf{Exercise 1 - Problem 1}}
%  \put(42, 784){\textbf{Ville R�is�nen}}
%  \put(445, 810){\textbf{Please, do not distribute!}}
}

\addtolength{\evensidemargin}{-1cm}
\addtolength{\oddsidemargin}{-1cm}
\addtolength{\marginparwidth}{-2.0cm}
\addtolength{\topmargin}{-2cm}
\addtolength{\textwidth}{3cm}
\addtolength{\textheight}{2cm}
\addtolength{\columnsep}{0.5cm}

 
\lhead {Spivak Comprehensive Introduction to Differential Geometry vol. 1 \\Ville R�is�nen}
\rhead {\today}
\thispagestyle{fancy}

\newtheorem {problem} {Problem}
\newtheorem {definition} {Definition}

\newcommand{\vu}[1]
{
	\mathbf{\hat #1}
}
\newcommand{\vf}[1]
{
	\mathbf{\vec #1}
}
\newcommand{\mc}[1]
{
	\mathcal{#1}
}
\newcommand{\vc}[1]
{
	\boldsymbol{#1}
}
\newcommand{\vcc}[1]
{
	\widetilde{\boldsymbol{#1}}
}
\newcommand{\cc}[1]
{
	\widetilde{#1}
}
\newcommand{\sgn}
{
  \textrm{sgn}\:
}
\newcommand{\pdiff}[2]
{
	\frac{\partial #1}{\partial #2}
}
\newcommand{\diff}[2]
{
	\frac{d #1}{d #2}
}
\newcommand{\vali}
{
  \begin {displaymath}\:\end{displaymath}
}
\newcommand{\norm}[2]
{
  ||#1||_{#2}
}

\newcommand{\solution}
{
   {\textbf{\textrm{Solution:}}}
}
\newcommand{\motivation}
{
  {\textbf{\textrm{Motivation:}}}
}

\begin {document}

\section{Chapter 1}

\begin {definition}
  A \textbf{metric} on a set $X$ is a function $d:X\times X\to\mathbb{R}$ with
  the following properties for all $x, y, z\in X$:
  \begin {eqnarray*}
    &1:& d(x, y) \geq 0 \\
   &2:& d(x, y) = 0 \Leftrightarrow x = y \\
    &3:&  d(x, y) = d(y, x) \\
   &4:& d(x, z) \leq d(x, y) + d(y, z).
  \end {eqnarray*}
  An \textbf{$\epsilon$-ball} centered at $x\in X$ in metric space $(X, d)$ is the set
  \begin {eqnarray}
    B_d(x, \epsilon) = \{y\in X : d(x, y) < \epsilon\}
  \end {eqnarray}
\end {definition}

\begin {flushleft}
\textbf{\textrm{Lemma 13.1 (Munkres):}} Left $X$ be a set; let $\mathcal B$ be
a basis for a topology $\mathcal{T}$ on $X$. Then $\mathcal{T}$ equals the collection
of all unions of elements of $\mathcal{B}$.
\end {flushleft}

\begin {definition}
  If $d$ is a metric on the set $X$, then the collection of all $\epsilon$-balls
  $B_d(x, \epsilon)$, for $x\in X$ and $\epsilon>0$, is a basis for a topology on $X$
  called the \textbf{metric topology} induced by $d$.
\end {definition}

\hrule

\begin {problem}
  Show that if d is a metric on X, then both $\overline d = d/(1+d)$ and
  $\overline d= \min(1,d)$ are also metrics and they are equivalent to d (i.e.., the identity
  map $1:(X, d)\to(X,\overline d)$ is a homeomorphism).
\end {problem}
\motivation{} On page 3, both definitions of
$\overline d$ provide a way given a metric $d$ to define an equivalent
(induces the same topology) metric bounded above by $1$. Metric bounded
above by $1$ allows then definition of a metric for disjoint union of two
metric spaces $(M_1, d_1)$ and $(M_2, d_2)$
\begin {displaymath}
  d(x,y) = \left\{
  \begin {array}{ll}
    \overline d_i(x,y) & \textrm{if there is some }i\: \textrm{such that }x,y\in M_i\\
    1, & otherwise.
  \end {array}
  \right.
\end {displaymath}
\begin {solution}\end {solution}
Let us first focus on the case $\overline d = d/(1+d)$:

Let $x, y, z\in X$.
Since $d$ is a metric $d(x, y) \geq 0$ and also $1 + d(x,y) \geq 0$. Therefore
\begin {displaymath}
  \overline d(x,y) = \dfrac{d(x,y)}{1+d(x,y)} \geq 0 \quad (1 : \textrm{non-negativity}).
\end {displaymath}
Let $\overline d(x,y)=0$. Then, since denominator is always non-zero, $d(x,y)=0$ must
hold. Since $d$ is a metric $x=y$ must hold (2: identity). Since $d$ is symmetric, simple
computation reveals
\begin {eqnarray*}
  \overline d(y, x) = \frac{d(y,x)}{1+d(y,x)} = \frac{d(x,y)}{1+d(x,y)}= \overline d(x,y) \quad ( 3: \textrm{symmetry}).
\end {eqnarray*}
To prove the triangle inequality, first note that if
$d(x,y) \geq d(x,z)$ or $d(y,z) \geq d(x,z)$, then since $\overline d$ is increasing
w.r.t. $d$, clearly 
\begin {eqnarray*}
  \overline d(x,z) \leq \overline d(x,y) + \overline d(y, z)
\end {eqnarray*}
holds. Suppose that $d(x, z) > d(x,y)$ and $d(x, z) > d(y, z)$. Application of this and
the triangle inequality for $d$, yields
\begin {eqnarray*}
  \overline d(x, z) &=&
  \frac{d(x, z)}{1 + d(x, z)} \\
  &\leq&
  \frac{d(x, y)}{1 + d(x, z)} + \frac{d(y, z)}{1 + d(x, z)} \\
  &\leq&
  \frac{d(x, y)}{1 + d(x, y)} + \frac{d(y, z)}{1 + d(y, z)} \\
  &=& \overline d(x,y) + \overline d(y, z) \quad (4 : \textrm{triangle inequality}).
\end {eqnarray*}
Therefore, $\overline d$ is a metric. We need to now show that the metrics $d$ and $\overline d$
are equivalent. That is, they must induce the same topology. The identity map 
$I: (X,d)\to (X, \overline d)$ is a homeomorphism if it and its inverse are
continuous. That is, $I^-1 U = U$ and $IV =V$ are open for
any open $U\subseteq(X, \overline d)$ and $V\subseteq(X, d)$ since $I$ and $I^{-1}$ are
continuous.

Let $U\subset(X,\overline d)$ and $V\subset(X, d)$ be open. Then $U$ and $V$
can be expressed as an union of $\epsilon$-balls.
Inverse image of each $\epsilon$-ball is the exactly same set but is assigned a different size
\begin {eqnarray*}
  \overline d(x, y) = \frac{d(x,y)}{1 + d(x,y)}\leq \epsilon
  &\Leftrightarrow&
  %d(x, y) (1 - \epsilon) \leq \epsilon \\
  %&\Leftrightarrow&
  d(x, y) \leq
  \frac{\epsilon}{1-\epsilon}.
\end {eqnarray*}
That is, for $\epsilon\in [0, \infty)$,
\begin {eqnarray*}
  I^{-1}B_{\overline d}(x, \epsilon) = B_d\left(x, \:\epsilon/(1-\epsilon)\right)
\end {eqnarray*}
holds. Similarly for any $\epsilon > 0$
\begin {eqnarray}
  IB_d(x, \epsilon) = B_{\overline d}\left(x, \:\epsilon/(1+\epsilon)\right).
\end {eqnarray}
Thus images and inverse images of $\epsilon$-balls by $I$ get mapped into $\epsilon$-balls
of different size. Since union of arbitrary family of open sets is open, $IV$ and
$I^{-1}U$ must be open. $\quad\square$

Let us now consider the case $\overline d=\min(1,d)$:

Let $x, y, z\in X$. Since $d(x,y)\geq 0$, $\overline d(x, y) = \min (d(x,y), 1)\geq 0$ must also hold
$( 1: $ non-negativity$)$. Let $\overline d(x,y) = 0$. Then clearly $d(x, y)=0$ and since $d$ is a
metric $x=y$ must hold $(2: $ identity$)$. Symmetry follows directly from $d$ when $d(x,y)\leq 1$.
Otherwise, $\overline d(y,x) = \overline d(x,y) = 1$.

To prove triangle inequality, consider first the case when $d(x,z) < 1$.
If $d(x, y) \geq 1$ or $d(y,z) \geq 1$, the triangle inequality is automatically satisfied.
Suppose then that $d(x,y) < 1$ and $d(y,z) < 1$ hold.
Then
\begin {eqnarray}
  \overline d(x, z) = d(x,z) \leq d(x,y) + d(y, z) = \overline d(x, y) + \overline d(y, z).
\end {eqnarray}
If $d(x,z) \geq 1$, then by triangle inequality for $d$, we must have 
$1 \leq d(x,z) \leq d(x, y) + d(y, z) $ and we obtain
\begin {eqnarray}
  1 = \overline d(x,z) \leq \overline d(x, y) + \overline d(y, z).
\end {eqnarray}
Thus, $\overline d$ is a metric.

Let $U\subseteq (X, d)$ be open, $x\in U$ and $I:X\to X$ the identity map.
Since $U$ is open w.r.t. $d$, there is $\epsilon>0$ so that $B_d(x, \epsilon)\subseteq U$.
Select $\epsilon':= max(\epsilon, 1/2)$. Then $B_{\overline d}(x, \epsilon')\subseteq U$.
Thus $U$ is open in $(X, \overline d)$. If $V\subseteq (X, \overline d)$ is open and
$y\in V$, then the exists $\epsilon>0$ such that $B_{\overline d}(x, \epsilon)\subseteq V$.
Note that, if $\epsilon \geq 1$, $B_{\overline d}(x, \epsilon)=X$ holds. However,
$B_d(x, \epsilon)\subseteq V$ follows always.  Therefore, the both metrics induce the same
topology.
$\quad\square$

\begin {problem}
  If $(X_i, d_i)$ are metric spaces, for $i\in I$, with metrics $d_i < 1$, and
  $X_i\cap X_j = \emptyset$ for $i\neq j$, then $(X, d)$ is a metric space, where
  $X = \cup_iX_i$ and $d(x, y) = d_i(x, y)$ if $x, y\in X_i$ for some $i$, while
  $d(x,y)=1$ otherwise. Each $X_i$ is an open subset of $X$, and $Y$ is homeomorphic
  to $X$ if and only if $Y=\cup_iY_i$ where the $Y_i$ are disjoint open sets and
  $Y_i$ is homeomorphic to $X_i$ for each $i$. The space $(X, d)$ (or any space
  homeomorphic to it) is called the \textbf{disjoint union} of the spaces $X_i$.
\end {problem}
\motivation{}In problem 1, we showed that for any metric $d$, we can define an
equivalent metric $\overline d$ limited to the interval $[0, 1)$. In this problem,
we will show that given a set of disjoint metric spaces, we can define a metric
and thus a metric topology for the union of the spaces using the construction of
the previous problem.

If the original metrics were used in the definition of the metric for union
$(X, d)$, there would be no trivial way to select distances between disjoint
subspaces: Since metric must be limited, one cannot simply select $d(x,y) = \infty$
when $x\in X_i, y\in X_j$ and $i\neq j$. Furthermore, other definitions might
lead to difficulties of neighborhoods of points intersecting other components.
The above construction satisfies two important properties:
\begin {itemize}
\item The topology of each component remains the same.
\item The distances within components are smaller than between components.
\end {itemize}
\begin {solution}\end{solution}
Let us first show that $d:X\times X\to[0, 1]$ is a metric:

Since each $d_i\geq 0$ and for point pairs with points from two different
components $d(x,y)=1$, we have $d\geq 0$ $(1:$ non-negativity$)$.
Let $x, y\in X$ so that $d(x,y)=0$. Then $x,y\in X_i$ for some $i$ and we
have $d_i(x,y)=0$. Since $d_i$ is a metric, $x=y$ must hold $(2:$ identity$)$.
Let $x,y\in X$. If $x$ and $y$ are members of different components,
$d(y,x)=d(x,y)=1$ must hold. Suppose $x, y\in X_i$ for some $i$. Then, since
$d_i$ is a metric,
\begin {eqnarray*}
  d(y,x) = d_i(y,x) = d_i(x,y) = d(x, y)
\end {eqnarray*}
must hold. $(3:$ symmetry$)$. Let $x,y,z\in X$. If all belong to the same
component $i$, triangle inequality follows from properties of metric $d_i$.
If all three belong to different components,
\begin {eqnarray*}
  d(x, z) = 1 \leq 2 = d(x, y) + d(y, z)
\end {eqnarray*}
and triangle inequality clearly holds.
Suppose then that $x,y\in X_i$ and $z\in X_j$, where $i\neq j$.
Then $d(x, z) \leq 1$ and $d(y, z)\leq 1$. Since also $d(x,y)\geq 0$,
triangle inequality follows. Thus, $d$ is a metric.

Each $X_i$ is open since 
\begin {eqnarray*}
  X_i = \bigcup_{x\in X_i}B_d(x, 1).
\end {eqnarray*}
and $(X_i, d_i)$ is topological space. Note that, each $X_i$ must be also
closed since $\cup_{j\neq i} X_j$ is open. Thus $X_i$ are the components of $X$.


Let us show that $Y$ is homeomorphic with $X$
if and only if $Y$ is disjoint union of open sets $Y_i$ homeomorphic with $X_i$
for each $i$: Suppose first that $f:X\to Y$ is homeomorphism.
Since $f^{-1}$ is continuous, the sets $X_i$ are disjoint and $f$ is bijection,
the sets $Y_i:= f(X_i)$ must be open and disjoint. Moreover restriction of $f$
to $X_i$ $f_i: X_i \to Y_i$ must be bijection. Topology for $Y_i$ is induced
by $f_i$. 

Suppose now that $Y$ is disjoint union of open sets $Y_i$ homeomorphic with $X_i$
for each $i$. Define
\begin {eqnarray*}
  f : X\to Y : \left.f\right|_{X_i} = Y_i.
\end {eqnarray*}
Then $f$ is clearly a bijection. Suppose $U\subseteq Y$ is open. Then $U$ can be
written as union of open sets limited to individual $Y_i$. Inverse image of each
such set is open in $X_i$. Thus, also inverse image of $U$ must be open and thus
$f$ is continuous. Same argument can be repeated for $f^{-1}$. Thus, $f$ must be
homeomorphism.
$\quad\square$


\end {document}
