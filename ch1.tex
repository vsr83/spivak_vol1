\documentclass [12pt, a4paper] {article}

%Babel tekee kuulemma jotain typer��, joten seuraava on v�ltt�m�t�nt�.
\usepackage[all, knot] {xy}
\let\oldxy\xy\def\xy{\begingroup\catcode`\"12\oldxy}
\let\oldendxy\endxy\def\endxy{\oldendxy\endgroup} 
\xyoption {arc}
\usepackage [latin1] {inputenc}
\usepackage [T1] {fontenc}
\usepackage [english] {babel}
\usepackage [dvips] {graphicx}
\usepackage {amsfonts}
\usepackage {amssymb}
\usepackage {amsmath}
\usepackage {verbatim}
\usepackage {fancyhdr}
\usepackage {psfrag}
\usepackage {listings}
\usepackage {eso-pic}
%\usepackage[urw-garamond]{mathdesign}
\usepackage[small,bf]{caption}
\usepackage{enumitem}
\usepackage{stix}
\usepackage{lipsum}

\newcommand{\BackgroundPic}{
%  \put(0,0){
%    \parbox[b][\paperheight]{\paperwidth}{
%      \vfill
%      \centering
%      \includegraphics[width=\paperwidth, height=\paperheight, keepaspectratio%]{
%kansilehti.ps}
%    }
%  }
%  \textbf{SMG-5156 - Electromagnetic Modelling I}\\
%  \textbf{Exercise 1 - Problem 4}\\
%  \textbf{Ville R�is�nen}
%\end {flushleft}

%  \put(42, 810){\textbf{SMG-5156 - Electromagnetic Modelling I}}
%  \put(42, 797){\textbf{Exercise 1 - Problem 1}}
%  \put(42, 784){\textbf{Ville R�is�nen}}
%  \put(445, 810){\textbf{Please, do not distribute!}}
}

\addtolength{\evensidemargin}{-1cm}
\addtolength{\oddsidemargin}{-1cm}
\addtolength{\marginparwidth}{-2.0cm}
\addtolength{\topmargin}{-2cm}
\addtolength{\textwidth}{3cm}
\addtolength{\textheight}{2cm}
\addtolength{\columnsep}{0.5cm}

 
\lhead {Spivak Comprehensive Introduction to Differential Geometry vol. 1 \\Ville R�is�nen}
\rhead {\today}
\thispagestyle{fancy}

\newtheorem {problem} {Problem}
\newtheorem {theorem} {Theorem}
\newtheorem {definition} {Definition}

\newcommand{\vu}[1]
{
	\mathbf{\hat #1}
}
\newcommand{\vf}[1]
{
	\mathbf{\vec #1}
}
\newcommand{\mc}[1]
{
	\mathcal{#1}
}
\newcommand{\vc}[1]
{
	\boldsymbol{#1}
}
\newcommand{\vcc}[1]
{
	\widetilde{\boldsymbol{#1}}
}
\newcommand{\cc}[1]
{
	\widetilde{#1}
}
\newcommand{\sgn}
{
  \textrm{sgn}\:
}
\newcommand{\pdiff}[2]
{
	\frac{\partial #1}{\partial #2}
}
\newcommand{\diff}[2]
{
	\frac{d #1}{d #2}
}
\newcommand{\vali}
{
  \begin {displaymath}\:\end{displaymath}
}
\newcommand{\norm}[2]
{
  ||#1||_{#2}
}

\newcommand{\solution}
{
   {\textbf{\textrm{Solution:}}}
}
\newcommand{\motivation}
{
  {\textbf{\textrm{Motivation:}}}
}

\begin {document}

\section{Chapter 1}

\begin {definition}
  A \textbf{metric} on a set $X$ is a function $d:X\times X\to\mathbb{R}$ with
  the following properties for all $x, y, z\in X$:
  \begin {eqnarray*}
    &1:& d(x, y) \geq 0 \\
   &2:& d(x, y) = 0 \Leftrightarrow x = y \\
    &3:&  d(x, y) = d(y, x) \\
   &4:& d(x, z) \leq d(x, y) + d(y, z).
  \end {eqnarray*}
  An \textbf{$\epsilon$-ball} centered at $x\in X$ in metric space $(X, d)$ is the set
  \begin {eqnarray}
    B_d(x, \epsilon) = \{y\in X : d(x, y) < \epsilon\}
  \end {eqnarray}
\end {definition}

\begin {flushleft}
\textbf{\textrm{Lemma 13.1 (Munkres):}} Left $X$ be a set; let $\mathcal B$ be
a basis for a topology $\mathcal{T}$ on $X$. Then $\mathcal{T}$ equals the collection
of all unions of elements of $\mathcal{B}$.
\end {flushleft}

\begin {definition}
  If $d$ is a metric on the set $X$, then the collection of all $\epsilon$-balls
  $B_d(x, \epsilon)$, for $x\in X$ and $\epsilon>0$, is a basis for a topology on $X$
  called the \textbf{metric topology} induced by $d$.
\end {definition}

\hrule

\begin {problem}
  Show that if $d$ is a metric on $X$, then both $\overline d = d/(1+d)$ and
  $\overline d= \min(1,d)$ are also metrics and they are equivalent to d (i.e.., the identity
  map $1:(X, d)\to(X,\overline d)$ is a homeomorphism).
\end {problem}
\motivation{} On page 3, both definitions of
$\overline d$ provide a way given a metric $d$ to define an equivalent
(induces the same topology) metric bounded above by $1$. Metric bounded
above by $1$ allows then definition of a metric for disjoint union of two
metric spaces $(M_1, d_1)$ and $(M_2, d_2)$
\begin {displaymath}
  d(x,y) = \left\{
  \begin {array}{ll}
    \overline d_i(x,y) & \textrm{if there is some }i\: \textrm{such that }x,y\in M_i\\
    1, & otherwise.
  \end {array}
  \right.
\end {displaymath}
\begin {solution}\end {solution}
Let us first focus on the case $\overline d = d/(1+d)$:

Let $x, y, z\in X$.
Since $d$ is a metric $d(x, y) \geq 0$ and also $1 + d(x,y) \geq 0$. Therefore
\begin {displaymath}
  \overline d(x,y) = \dfrac{d(x,y)}{1+d(x,y)} \geq 0 \quad (1 : \textrm{non-negativity}).
\end {displaymath}
Let $\overline d(x,y)=0$. Then, since denominator is always non-zero, $d(x,y)=0$ must
hold. Since $d$ is a metric $x=y$ must hold (2: identity). Since $d$ is symmetric, simple
computation reveals
\begin {eqnarray*}
  \overline d(y, x) = \frac{d(y,x)}{1+d(y,x)} = \frac{d(x,y)}{1+d(x,y)}= \overline d(x,y) \quad ( 3: \textrm{symmetry}).
\end {eqnarray*}
To prove the triangle inequality, first note that if
$d(x,y) \geq d(x,z)$ or $d(y,z) \geq d(x,z)$, then since $\overline d$ is increasing
w.r.t. $d$, clearly 
\begin {eqnarray*}
  \overline d(x,z) \leq \overline d(x,y) + \overline d(y, z)
\end {eqnarray*}
holds. Suppose that $d(x, z) > d(x,y)$ and $d(x, z) > d(y, z)$. Application of this and
the triangle inequality for $d$, yields
\begin {eqnarray*}
  \overline d(x, z) &=&
  \frac{d(x, z)}{1 + d(x, z)} \\
  &\leq&
  \frac{d(x, y)}{1 + d(x, z)} + \frac{d(y, z)}{1 + d(x, z)} \\
  &\leq&
  \frac{d(x, y)}{1 + d(x, y)} + \frac{d(y, z)}{1 + d(y, z)} \\
  &=& \overline d(x,y) + \overline d(y, z) \quad (4 : \textrm{triangle inequality}).
\end {eqnarray*}
Therefore, $\overline d$ is a metric. We need to now show that the metrics $d$ and $\overline d$
are equivalent. That is, they must induce the same topology. The identity map 
$I: (X,d)\to (X, \overline d)$ is a homeomorphism if it and its inverse are
continuous. That is, $I^-1 U = U$ and $IV =V$ are open for
any open $U\subseteq(X, \overline d)$ and $V\subseteq(X, d)$ since $I$ and $I^{-1}$ are
continuous.

Let $U\subset(X,\overline d)$ and $V\subset(X, d)$ be open. Then $U$ and $V$
can be expressed as an union of $\epsilon$-balls.
Inverse image of each $\epsilon$-ball is the exactly same set but is assigned a different size
\begin {eqnarray*}
  \overline d(x, y) = \frac{d(x,y)}{1 + d(x,y)}\leq \epsilon
  &\Leftrightarrow&
  %d(x, y) (1 - \epsilon) \leq \epsilon \\
  %&\Leftrightarrow&
  d(x, y) \leq
  \frac{\epsilon}{1-\epsilon}.
\end {eqnarray*}
That is, for $\epsilon\in [0, \infty)$,
\begin {eqnarray*}
  I^{-1}B_{\overline d}(x, \epsilon) = B_d\left(x, \:\epsilon/(1-\epsilon)\right)
\end {eqnarray*}
holds. Similarly for any $\epsilon > 0$
\begin {eqnarray}
  IB_d(x, \epsilon) = B_{\overline d}\left(x, \:\epsilon/(1+\epsilon)\right).
\end {eqnarray}
Thus images and inverse images of $\epsilon$-balls by $I$ get mapped into $\epsilon$-balls
of different size. Since union of arbitrary family of open sets is open, $IV$ and
$I^{-1}U$ must be open. $\quad\square$

Let us now consider the case $\overline d=\min(1,d)$:

Let $x, y, z\in X$. Since $d(x,y)\geq 0$, $\overline d(x, y) = \min (d(x,y), 1)\geq 0$ must also hold
$( 1: $ non-negativity$)$. Let $\overline d(x,y) = 0$. Then clearly $d(x, y)=0$ and since $d$ is a
metric $x=y$ must hold $(2: $ identity$)$. Symmetry follows directly from $d$ when $d(x,y)\leq 1$.
Otherwise, $\overline d(y,x) = \overline d(x,y) = 1$.

To prove triangle inequality, consider first the case when $d(x,z) < 1$.
If $d(x, y) \geq 1$ or $d(y,z) \geq 1$, the triangle inequality is automatically satisfied.
Suppose then that $d(x,y) < 1$ and $d(y,z) < 1$ hold.
Then
\begin {eqnarray}
  \overline d(x, z) = d(x,z) \leq d(x,y) + d(y, z) = \overline d(x, y) + \overline d(y, z).
\end {eqnarray}
If $d(x,z) \geq 1$, then by triangle inequality for $d$, we must have 
$1 \leq d(x,z) \leq d(x, y) + d(y, z) $ and we obtain
\begin {eqnarray}
  1 = \overline d(x,z) \leq \overline d(x, y) + \overline d(y, z).
\end {eqnarray}
Thus, $\overline d$ is a metric.

Let $U\subseteq (X, d)$ be open, $x\in U$ and $I:X\to X$ the identity map.
Since $U$ is open w.r.t. $d$, there is $\epsilon>0$ so that $B_d(x, \epsilon)\subseteq U$.
Select $\epsilon':= max(\epsilon, 1/2)$. Then $B_{\overline d}(x, \epsilon')\subseteq U$.
Thus $U$ is open in $(X, \overline d)$. If $V\subseteq (X, \overline d)$ is open and
$y\in V$, then the exists $\epsilon>0$ such that $B_{\overline d}(x, \epsilon)\subseteq V$.
Note that, if $\epsilon \geq 1$, $B_{\overline d}(x, \epsilon)=X$ holds. However,
$B_d(x, \epsilon)\subseteq V$ follows always.  Therefore, the both metrics induce the same
topology.
$\quad\square$

\begin {problem}
  If $(X_i, d_i)$ are metric spaces, for $i\in I$, with metrics $d_i < 1$, and
  $X_i\cap X_j = \emptyset$ for $i\neq j$, then $(X, d)$ is a metric space, where
  $X = \cup_iX_i$ and $d(x, y) = d_i(x, y)$ if $x, y\in X_i$ for some $i$, while
  $d(x,y)=1$ otherwise. Each $X_i$ is an open subset of $X$, and $Y$ is homeomorphic
  to $X$ if and only if $Y=\cup_iY_i$ where the $Y_i$ are disjoint open sets and
  $Y_i$ is homeomorphic to $X_i$ for each $i$. The space $(X, d)$ (or any space
  homeomorphic to it) is called the \textbf{disjoint union} of the spaces $X_i$.
\end {problem}
\motivation{}In problem 1, we showed that for any metric $d$, we can define an
equivalent metric $\overline d$ with range limited to the interval $[0, 1)$. In this problem,
we will show that given a set of disjoint metric spaces, we can define a metric
and thus a metric topology for the union of the spaces using the construction of
the previous problem.

If the original metrics were used in the definition of the metric for union
$(X, d)$, there would be no trivial way to select distances between disjoint
subspaces: Since metric must be limited, one cannot simply select $d(x,y) = \infty$
when $x\in X_i, y\in X_j$ and $i\neq j$. Furthermore, other definitions might
lead to difficulties of neighborhoods of points intersecting other components.
The above construction satisfies two important properties:
\begin {itemize}
\item The topology of each component remains the same.
\item The distances within components are smaller than between components.
\end {itemize}
\begin {solution}\end{solution}
Let us first show that $d:X\times X\to[0, 1]$ is a metric:

Since each $d_i\geq 0$ and for point pairs with points from two different
components $d(x,y)=1$, we have $d\geq 0$ $(1:$ non-negativity$)$.
Let $x, y\in X$ so that $d(x,y)=0$. Then $x,y\in X_i$ for some $i$ and we
have $d_i(x,y)=0$. Since $d_i$ is a metric, $x=y$ must hold $(2:$ identity$)$.
Let $x,y\in X$. If $x$ and $y$ are members of different components,
$d(y,x)=d(x,y)=1$ must hold. Suppose $x, y\in X_i$ for some $i$. Then, since
$d_i$ is a metric,
\begin {eqnarray*}
  d(y,x) = d_i(y,x) = d_i(x,y) = d(x, y)
\end {eqnarray*}
must hold. $(3:$ symmetry$)$. Let $x,y,z\in X$. If all belong to the same
component $i$, triangle inequality follows from properties of metric $d_i$.
If all three belong to different components,
\begin {eqnarray*}
  d(x, z) = 1 \leq 2 = d(x, y) + d(y, z)
\end {eqnarray*}
and triangle inequality clearly holds.
Suppose then that $x,y\in X_i$ and $z\in X_j$, where $i\neq j$.
Then $d(x, z) \leq 1$ and $d(y, z)\leq 1$. Since also $d(x,y)\geq 0$,
triangle inequality follows. Thus, $d$ is a metric.

Each $X_i$ is open since 
\begin {eqnarray*}
  X_i = \bigcup_{x\in X_i}B_d(x, 1).
\end {eqnarray*}
and $(X_i, d_i)$ is topological space. Note that, each $X_i$ must be also
closed since $\cup_{j\neq i} X_j$ is open.
%Thus $X_i$ are the components of $X$.

Let us show that $Y$ is homeomorphic with $X$
if and only if $Y$ is disjoint union of open sets $Y_i$ homeomorphic with $X_i$
for each $i$: Suppose first that $f:X\to Y$ is homeomorphism.
Since $f^{-1}$ is continuous, the sets $X_i$ are disjoint and $f$ is bijection,
the sets $Y_i:= f(X_i)$ must be open and disjoint. Moreover restriction of $f$
to $X_i$ $f_i: X_i \to Y_i$ must be a bijection. Topology for $Y_i$ is induced
by $f_i$. 

Suppose now that $Y$ is disjoint union of open sets $Y_i$ homeomorphic with $X_i$
for each $i$. Define
\begin {eqnarray*}
  f : X\to Y : \left.f\right|_{X_i} = Y_i.
\end {eqnarray*}
Then $f$ is clearly a bijection. Suppose $U\subseteq Y$ is open. Then $U$ can be
written as union of open sets limited to individual $Y_i$. Inverse image of each
such set is open in $X_i$. Thus, also inverse image of $U$ must be open and thus
$f$ is continuous. Same argument can be repeated for $f^{-1}$. Thus, $f$ must be
homeomorphism.
$\quad\square\\$
\hrule
\begin {definition}
  \label{def:manifold}
  A topological space $M$ is \textbf{locally Euclidean} if for each $x\in M$ there is
  a neighborhood $U$ of x and some integer $n\geq 0$ so that $U$ is homeomorphic
  to $\mathbb{R}^n$. A \textbf{topological manifold} is a metric space $M$ that is locally
  Euclidean.
\end {definition}
The rather surprising definition as a metric space is made more clear by the
following two theorems.
\begin {theorem}
  If $X$ is connected locally compact metric space, then $X$ is $\sigma$-compact.
\end {theorem}
\begin {theorem}
  \label{th:second_countable}
  For any locally Euclidean Hausdorff space, the following conditions are equivalent:
  \begin {enumerate}
  \item Each component of $M$ is $\sigma$-compact,
  \item Each component of $M$ is second contable (has a countable base for the topology).
  \item $M$ is metrizable.
  \item $M$ is paracompact.
  \end {enumerate}  
\end {theorem}
Since every metric space Hausdorff, by Definition \ref{def:manifold}
and Theorem \ref{th:second_countable}, any manifold must be second-countable. Thus, we
obtain a link to a more common definition of topological manifolds as second-countable
and locally Euclidean Hausdorff spaces. 

\begin {definition}
  A topological space $X$ is said to be \textbf{locally compact at $x$} if there is some compact
  subspace $C$ of $X$ that contains neighborhood of $x$. If $X$ is compact at each
  $x\in X$, $X$ is \textbf{locally compact}.
\end {definition}

\begin {definition}
  In topological space $X$ a \textbf{path between $x\in X$ and $y\in X$} is a continuous map
  $f:[a,b]\to X$ so that $f(a)=x$ and $f(b)=y$ hold. Topological space $X$ is \textbf{path connected}
  if there is a path between every pair of points of $X$.
  A topological space $X$ is said to be \textbf{locally path connected at $x$} if every
  neighborhood $U$ of $x$ contains a path conncted neighborhood $V$ of $x$.
  An \textbf{arc} is an injective path and $X$ is \textbf{arcwise connected} if each
  pair of points can be connected by an arc.
\end {definition}
\hrule
$\\$

\begin {problem}
  $\:$
  \begin {enumerate}
  \item Every manifold is locally compact.
  \item Every manifold is locally pathwise connected, and a connected manifold
    pathwise connected.
  \item A connected manifold is arcwise connected.
  \end {enumerate}
\end {problem}
\begin {solution}\end {solution} Let $X$ be a manifold and let us first show that
$X$ is locally compact.

Let $x\in X$. Since $X$ is locally Euclidean, there exists a neighborhood $U$ of
$x$ and a homeomorphism $f:U\to\mathbb{R}^n$. Since $\mathbb{R}^n$ is locally
compact, there exists compact subspace $V\subseteq\mathbb{R}^n$, which contains
a neighborhood $W$ of $f(x)$. Since $f$ is continuous, $f^{-1}(V)$ is a compact
subspace of $X$ and contains an neighborhood $f^{-1}(W)$ of $x$. Thus $X$ is
locally compact. $\quad\square$

Let us now show that $X$ is locally path-connected.
Let $x\in X$. Since $X$ is locally Euclidean, there exists a neighborhood $U$ of
$x$ and a homeomorphism $f:U\to\mathbb{R}^n$. Since $\mathbb{R}^n$ is locally
path-connected, and $f(U)$ is neighborhood of $f(x)$, there is path connected
neighborhood $W$ contained in $f(U)$. Since $f^{-1}$ is continuous,
$v:= f^{-1}(W)\subseteq U$ is path connected neighborhood of $x$.
Thus, $X$ is locally path connected. $\quad\square$

Suppose now that $X$ is a connected manifold and let us show that $X$ is
pathwise connected: Let $x_0\in X$ and define
\begin {eqnarray}
  V := \{x\in X: \exists\: \textrm{path } x_0\to x\}
\end {eqnarray}
If $V$ is both open and closed, it must be equal to $X$ since $X$ has only one
component.
Let $x\in V$. Since $X$ is open and locally path-connected, there must be a neighborhood
of $x$ in $X$, which contains a path connected neighborhood $W$ of $x$. Since there
exists a path from $x_0$ to $x$ and $W$ is path connected, there must also exist a
path from $x_0$ to every $y\in W$. Thus, $V$ is open.

Let $x\in X\setminus V$. Then, there is no path $x_0\to x$.
Suppose every neighborhood of $x$ intersects with $V$. Then, since $X$ is locally
connected there must be path from $x$ to a point of $V$. This implies that $x\in V$,
which is a contradiction. Thus, there must an neighborhood of $x$ contained in
$X\setminus V$. Thereforce, $X\setminus V$ must be open and $V$ closed. $\quad\square$

Let $X$ be connected manifold. Let us show that $X$ is arcwise connected.
The proof for local path connectedness works just as well for arcwise connectedness
since every $\epsilon$-ball of $\mathbb{R}^n$ is clearly arcwise connected.
Any $x\in X$ has a neighborhood in $U\subseteq X$ and a homeomorphismm
$f:U\to\mathbb{R}^n$.
Inverse image of $\epsilon$-ball neighborhood of $f(x)$ is arcwise connected
neighborhood contained in the above neighborhood. Thus the proof for path
connectedness can be easily modified for arcwise connectedness. $\quad\square$

\begin {problem}
  A space $X$ is called locally connected if for each $x\in X$ it is the case that
  every neighborhood of $x$ contains a connected neighborhood.

  \begin {enumerate}[label=(\alph*)]
  \item Connectedness does not imply local connectedness.
  \item An open subset of a locally connected space is locally connected.
  \item $X$ is locally connected if and only if components of open sets are open,
    so every neighborhood of a point in a locally connected space contains an
    open connected neighborhood.
  \item A locally connected space is homeomorphic to the disjoint union of its
    components.
  \item Every manifold is locally connected, and consequently homeomorphic to the
    disjoint union of its components, which are open submanifolds.
  \end {enumerate}
\end {problem}
\begin {solution}\end {solution}\textbf{(a)}
Consider the \emph{Topologist's sine curve:}
\begin {eqnarray*}
  M = \{
  (x, \sin 1/x) : x\in (0, 1]\}
  \subseteq\mathbb{R}^2
\end {eqnarray*}
with the closure
\begin {eqnarray*}
  \overline M = \{
  (x, \sin 1/x) : x\in (0, 1]\} \cup \{(0,y): y\in[0,1]\}
  \subseteq\mathbb{R}^2
\end {eqnarray*}
and subspace topology inherited from $\mathbb{R}^2$
\begin {eqnarray*}
  \tau = \{U\cap M : U\subseteq\:\mathbb{R}^2 \:\textrm{open}\}.
\end {eqnarray*}
Since $M$ is the image of an connected set by a continuous function, it is
connected. Moreover, closure of an connected set is connected.

Select $U:= B_d((0,0), \epsilon)$, where $\epsilon < 1$.
Every neighborhood $B_d((0,0), \epsilon)\cap M$ contains infinite number
of components of $M$. $\quad\square$

\textbf{(b)} Let $M$ be locally connected and $U\subseteq M$ be open.
Now let $x\in U$, and $V\subseteq U$ open neighborhood of $x$.
Since $U\cap V$ is neighborhood of $x$ in $M$, there is connected neighborhood
$W\subseteq U\cap V$ of $x$. $\quad\square$

\textbf{(c)} Let $M$ be locally connected and $U\subseteq M$ be open.
Note that now, $U$ may consist of arbitrary number of components.
Let $C\subseteq U$ be a component of $U$ and $x\in C$. Since $M$ is locally connected,
there is a connected neighborhood $V\subseteq M$ of $x$. Since $C$ is
connected and contains $x$, $V\subseteq C$ must hold. Since $x\in C$
is arbitrary, $C$ must be open.

Suppose now that every component of each open $U\subseteq M$ is open.
Let $x\in M$ and $U\subseteq M$ be a neighborhood of $x$.
Let $C$ be the component of $U$ that contains $x\in M$. Then $C$ is
open and thus a connected neighborhood of $x$. Thus, $M$ is locally
connected. $\quad\square$

\textbf{(d)} Let $M$ be locally connected. 
Let $C_\alpha : \alpha\in I$ be the family of all components of $M$ and
$f_\alpha: C_\alpha \to M$ canonical injections of $C_\alpha$ to $M$.
Then \textbf{disjoint union topology} is the topology, in which
$U\subseteq\bigcup_\alpha C_\alpha$ is open if and only if $f_\alpha^{-1}(U)$ is open
for all $\alpha\in I$.
Define
\begin {eqnarray*}
  f: \bigcup_{\alpha\in I}C_\alpha \to M : \left.f\right|_{C_\alpha} = f_\alpha.
\end {eqnarray*}
Then since $M = \bigcup_\alpha C_\alpha$, $f$ must be a bijection.
Since $M$ is open and locally connected, it follows from $(c)$ that every component
$C_\alpha\subseteq M$ must be open. 

Now if $U\subseteq M$ is open, it can be written as a disjoint union of connected
open sets and $f_\alpha^{-1}(U)$ is either empty or an non-empty open set. Thus,
$U$ is open in $\bigcup_\alpha C_\alpha$. If $U$ is open in $\bigcup_\alpha C_\alpha$,
$f_\alpha^{-1}(U)$ is open for all $\alpha\in I$. Since $U$ is an union of open sets
in $M$, $U$ must be open in $M$. $\quad\square$

\textbf{(e)} Follows from $(b),(c),(d),(e)$. $\quad\square$
\newpage

\begin {theorem} (Invariance of Domain)
  If $U\subseteq\mathbb{R}^n$ is open and $f:U\to\mathbb{R}^n$ is one-to-one and
  continuous, then $f(U)\subseteq\mathbb{R}^n$ is open.
\end {theorem}

\begin {problem}
  $\:$
  
  \begin {enumerate}[label=(\alph*)]
  \item The neighborhood $U$ in our definition of a manifold is always open.
  \item The integer $n$ in our definition is unique for each $x\in M$.
  \end {enumerate}  
\end {problem}
\begin {solution}\end{solution}
\textbf{(a)} The following argument is incorrect:
\begin {quote}
Let $U\subseteq M$ and $f:U\to\mathbb{R}^n$ homeomorphism. Let
us show that $U$ is always open. Since $f$ is continuous and $\mathbb{R}^n$
is open $f^{-1}(\mathbb{R}^n) = U$ must be open.
\end {quote}
\textbf{It follows that $U$
is open in $U$ but it does not necessarily follow that $U$ is open in $M$!}

Before accepting conclusion of this theorem, Spivak considers \textbf{neighborhood}
$U\subseteq M$ of $x\in M$ to be a set, which contains an open subset $V\subset U$
that contains $x\in M$.

Let $U\subseteq M$ neighborhood of $x\in M$.
There is a homeomorphism $f : U \to \mathbb{R}^n$.
Since $M$ is a locally Euclidean, each point $x\in U$ has also an open neighborhood
and each open neighborhood contains some $\epsilon$-ball $V_x$ of $x$, which is trivially
homeomorphic with $\mathbb{R}^n$ via some $f_x:V_x\to\mathbb{R}^n$.
Each $V_x\cap U$ is open in subspace topology of $U$ but not necessarily in $M$.

Since $f$ is homeomorphism, $f(V_x\cap U)$ is open in $\mathbb{R}^n$ and contains $f(x)$.
There must be a $B:= B_d(f(x), \epsilon)$ contained in $f(V_x\cap U)$.
Now $\left.f_x\circ f^{-1}\right|_B:B\to\mathbb{R}^n$ must be an injection and continuous.
Therefore, according to Invariance of Domain, $f_x\circ f^{-1}(B)\subseteq\mathbb{R}^n$
must be open. Then also $f_x^{-1}(f_x(f^{-1}(V_x))) = f^{-1}(B)\subseteq V_x\cap U$
must be open subset of $M$ containing $x$. Since $x\in U$ was arbitrary, $U$ must
be open. $\quad\square$

\textbf{(b)} Let $x\in M$ and suppose there is $B(x, \epsilon)$ for some $\epsilon > 0$
homeomorphic with both $\mathbb{R}^n$ and $\mathbb{R}^m$, where $m\neq n$. Then it follows
that $\mathbb{R}^n$ and $\mathbb{R}^m$ are homeomorphic, which is a contradiction.
$\quad\square$
\end {document}
