\documentclass [12pt, a4paper] {article}

%Babel tekee kuulemma jotain typer��, joten seuraava on v�ltt�m�t�nt�.
\usepackage[all, knot] {xy}
\let\oldxy\xy\def\xy{\begingroup\catcode`\"12\oldxy}
\let\oldendxy\endxy\def\endxy{\oldendxy\endgroup} 
\xyoption {arc}
\usepackage [latin1] {inputenc}
\usepackage [T1] {fontenc}
\usepackage [english] {babel}
\usepackage [dvips] {graphicx}
\usepackage {amsfonts}
\usepackage {amssymb}
\usepackage {amsmath}
\usepackage {verbatim}
\usepackage {fancyhdr}
\usepackage {psfrag}
\usepackage {listings}
\usepackage {eso-pic}

%\usepackage[urw-garamond]{mathdesign}
\usepackage[small,bf]{caption}

\newcommand{\BackgroundPic}{
%  \put(0,0){
%    \parbox[b][\paperheight]{\paperwidth}{
%      \vfill
%      \centering
%      \includegraphics[width=\paperwidth, height=\paperheight, keepaspectratio%]{
%kansilehti.ps}
%    }
%  }
%  \textbf{SMG-5156 - Electromagnetic Modelling I}\\
%  \textbf{Exercise 1 - Problem 4}\\
%  \textbf{Ville R�is�nen}
%\end {flushleft}

%  \put(42, 810){\textbf{SMG-5156 - Electromagnetic Modelling I}}
%  \put(42, 797){\textbf{Exercise 1 - Problem 1}}
%  \put(42, 784){\textbf{Ville R�is�nen}}
%  \put(445, 810){\textbf{Please, do not distribute!}}
}

\addtolength{\evensidemargin}{-1cm}
\addtolength{\oddsidemargin}{-1cm}
\addtolength{\marginparwidth}{-2.0cm}
\addtolength{\topmargin}{-2cm}
\addtolength{\textwidth}{3cm}
\addtolength{\textheight}{2cm}
\addtolength{\columnsep}{0.5cm}

 
\lhead {Spivak Comprehensive Introduction to Differential Geometry vol. 1 \\Ville R�is�nen}
\rhead {\today}
\thispagestyle{fancy}

\newtheorem {problem} {Problem}
\newtheorem {definition} {Definition}

\newcommand{\vu}[1]
{
	\mathbf{\hat #1}
}
\newcommand{\vf}[1]
{
	\mathbf{\vec #1}
}
\newcommand{\mc}[1]
{
	\mathcal{#1}
}
\newcommand{\vc}[1]
{
	\boldsymbol{#1}
}
\newcommand{\vcc}[1]
{
	\widetilde{\boldsymbol{#1}}
}
\newcommand{\cc}[1]
{
	\widetilde{#1}
}
\newcommand{\sgn}
{
  \textrm{sgn}\:
}
\newcommand{\pdiff}[2]
{
	\frac{\partial #1}{\partial #2}
}
\newcommand{\diff}[2]
{
	\frac{d #1}{d #2}
}
\newcommand{\vali}
{
  \begin {displaymath}\:\end{displaymath}
}
\newcommand{\norm}[2]
{
  ||#1||_{#2}
}

\newcommand{\solution}
{
  \flushleft{\textbf{\textrm{Solution:}}}
}
\newcommand{\motivation}
{
  \flushleft{\textbf{\textrm{Motivation:}}}
}

\begin {document}

\section{Chapter 1}

\begin {definition}
  A \textbf{metric} on a set $X$ is a function $d:X\times X\to\mathbb{R}$ with
  the following properties for all $x, y, z\in X$:
  \begin {eqnarray*}
    &1:& d(x, y) \geq 0 \\
   &2:& d(x, y) = 0 \Leftrightarrow x = y \\
    &3:&  d(x, y) = d(y, x) \\
   &4:& d(x, z) \leq d(x, y) + d(y, z).
  \end {eqnarray*}
  An \textbf{$\epsilon$-ball} centered at $x\in X$ in metric space $(X, d)$ is the set
  \begin {eqnarray}
    B_d(x, \epsilon) = \{y\in X : d(x, y) < \epsilon\}
  \end {eqnarray}
\end {definition}

\flushleft{\textbf{\textrm{Lemma 13.1 (Munkres):}}} Left $X$ be a set; let $\mathcal B$ be
a basis for a topology $\mathcal{T}$ on $X$. Then $\mathcal{T}$ equals the collection
of all unions of elements of $\mathcal{B}$.

\begin {definition}
  If $d$ is a metric on the set $X$, then the collection of all $\epsilon$-balls
  $B_d(x, \epsilon)$, for $x\in X$ and $\epsilon>0$, is a basis for a topology on $X$
  called the \textbf{metric topology} induced by $d$.
\end {definition}

\hrule

\begin {problem}
  Show that if d is a metric on X, then both $\overline d = d/(1+d)$ and
  $\overline d= \min(1,d)$ are also metrics and they are equivalent to d (i.e.., the identity
  map $1:(X, d)\mapsto(X,\overline d)$ is a homeomorphism).
\end {problem}

\motivation{} On page 3, both definitions of
$\overline d$ provide a way given a metric $d$ to define an equivalent
(induces the same topology) metric bounded above by $1$. Metric bounded
above by $1$ allows then definition of a metric for disjoint union of two
metric spaces $(M_1, d_1)$ and $(M_2, d_2)$
\begin {displaymath}
  d(x,y) = \left\{
  \begin {array}{ll}
    \overline d_i(x,y) & \textrm{if there is some }i \textrm{such that }x,y\in M_i\\
    1, & otherwise.
  \end {array}
  \right.
\end {displaymath}
\begin {solution}\end {solution}
Let us first focus on the case $\overline d = d/(1+d)$:

Let $x, y, z\in X$.
Since $d$ is a metric $d(x, y) \geq 0$ and also $1 + d(x,y) \geq 0$. Therefore
\begin {displaymath}
  \overline d(x,y) = \dfrac{d(x,y)}{1+d(x,y)} \geq 0 \quad (1 : \textrm{non-negativity}).
\end {displaymath}
Let $\overline d(x,y)=0$. Then, since denominator is always non-zero, $d(x,y)=0$ must
hold. Since $d$ is a metric $x=y$ must hold (2: identity). Since $d$ is symmetric, simple
computation reveals
\begin {eqnarray*}
  \overline d(y, x) = \frac{d(y,x)}{1+d(y,x)} = \frac{d(x,y)}{1+d(x,y)}= \overline d(x,y) \quad ( 3: \textrm{symmetry}).
\end {eqnarray*}
To prove the triangle inequality, first note that if
$d(x,y) \geq d(x,z)$ or $d(y,z) \geq d(x,z)$, then since $\overline d$ is increasing
w.r.t. $d$, clearly 
\begin {eqnarray*}
  \overline d(x,z) \leq \overline d(x,y) + \overline d(y, z)
\end {eqnarray*}
holds. Suppose that $d(x, z) > d(x,y)$ and $d(x, z) > d(y, z)$. Application of this and
the triangle inequality for $d$, yields
\begin {eqnarray*}
  \overline d(x, z) &=&
  \frac{d(x, z)}{1 + d(x, z)} \\
  &\leq&
  \frac{d(x, y)}{1 + d(x, z)} + \frac{d(y, z)}{1 + d(x, z)} \\
  &\leq&
  \frac{d(x, y)}{1 + d(x, y)} + \frac{d(y, z)}{1 + d(y, z)} \\
  &=& \overline d(x,y) + \overline d(y, z) \quad (4 : \textrm{triangle inequality}).
\end {eqnarray*}
Therefore, $\overline d$ is a metric. We need to now show that the metrics $d$ and $\overline d$
are equivalent. That is, they must induce the same topology. If the identity map 
$I: (X,d)\mapsto (X, \overline d)$ is a homeomorphism, $I^-1 U = U$ and $IV =V$ are open for
any open $U\subseteq(X, \overline d)$ and $V\subseteq(X, d)$.



\end {document}
